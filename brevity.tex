\chapter{Brevity}
\label{brevity}

Like most professions, combat aviation has its own jargon.
Called \emph{brevity}, it's incomprehensibly dense to a newcomer,
but it lets pilots communicate quickly in the heat of battle.

Most folks you find online will be using \textsc{us/nato} brevity.
There's many,
many terms,\punckern\footnote{An early-2000s version of an official
multiservice \textsc{us} training guide is declassified and freely available
online at \url{https://apps.dtic.mil/dtic/tr/fulltext/u2/a404426.pdf}}
but we'll cover some of the most common ones.
Before we dive in, know two things:
\begin{enumerate}
\item If you can't think of a term you want,
    \emph{use plain language!}
    The main goal is to be understood, quickly.
    Sounding cool is just an added bonus.
\item Some terms have one version for friendly forces
    and another for hostiles or unknown forces.
    Don't mix them up!
    If you're \emph{visual} on your wingman,
    you can fly in formation with them.
    But if you \emph{tally} your wingman,
    they'll wonder why you're trying to shoot them.
\end{enumerate}

\section{Who? Where?}

\begin{description}
\item[Capture] I see a ground target on my sensors (such as radar
    or targeting pod).

\item[Contact] I see a potential target on my sensors (usually radar).
    Also brevity for the potential target itself.
    Provide a location and additional information if you can.

    ``Contact, Bullseye 120 35, 18 thousand. Hot.''

\item[Declare] Tell me if the contact at the given location
    (using BRAA or a Bullseye call) is friendly or not (see below).
    Requested from command and control \acronym{(c2)} assets,
    like an airborne \designation{E-3} Sentry
    or a ground-based fighter controller.

\item[Friendly/Bogey/Outlaw/Bandit/Hostile]
    Air contacts, by increasing threat:
    \begin{description}
    \item[Friendly] An positively-identified allied contact.
    \item[Bogey] An unknown contact.
    \item[Outlaw] A bogey that is likely a hostile
        (e.g., that launched from an enemy airfield)
    \item[Bandit] An enemy contact---does not imply permission to engage.
    \item[Hostile] An enemy contact---free to fire.
    \end{description}

\item[BRAA] Short for \emph{Bearing, Range, Altitude, Aspect}.
    From my plane, the thing I'm describing is in this direction
    (bearing), for this distance (range), at this altitude.
    Provide the direction they're moving (aspect) if possible.
    Aspects:
    \begin{description}
    \item[Hot] heading towards you
    \item[Dragging] heading away from you
    \item[Beaming] heading perpendicular to you (provide a direction)
    \end{description}

    ``Bogey, BRAA 35 for 50, 18 thousand, hot.''

\item[Bullseye] The thing I'm describing is this direction and distance
    from the \emph{bullseye}.
    The bullseye is a preplanned location---it could be a large ground target
    (like an airfield), some landmark,
    or just an arbitrary point---what matters is that everyone knows where
    it is so they can use it as a shared reference.

    ``Engage armored column, Bullseye 180 20''

\item[Group] Air contacts flying together.

\item[Picture] Tell me about nearby contacts (with BRAA or Bullseye calls).
    Requested from \acronym{c2} assets.

\item[Pop-up] A new contact has suddenly appeared at the given location.

\item[Tally/No Joy] I see the enemy.\,/\,I don't see the enemy.

\item[Tumbleweed] I've lost situational awareness \acronym{(sa)}
    and don't know what's going on.
    Help me get reoriented, if you can.

\item[Visual/Blind] I see the friendly.\,/\,I don't see the friendly.
\end{description}

\section{Administration}

What are you doing? What's your status?
\begin{description}

\item[As Fragged] I'm continuing as-planned.

\item[Bingo] I'm low on fuel and need to leave right now if I want enough
    to land.

\item[Commit] Intercept the given group.

    You usually brief a \emph{commit range}---if contacts get within that range,
    you stop what you're doing and commit on those contacts.

\item[Fence In] Get ready for combat---shut off external lights,
    arm weapons, enable electronic countermeasures \acronym{(ecm)}, etc.
    From ``crossing the fence'' as you enter bad guy territory.
    Once you're out of combat, you \emph{fence out}.

\item[Gate/Buster] Fly as fast as you can [with/without] afterburner.

\item[Joker] I'm starting to run low on fuel and should think about getting
    out of here. Comes before bingo.

\item[Rolex] Delay the (package's) plan by the given amount of time.

    ``Package, rolex three minutes.''
\end{description}

\section{Combat}

Shooting things and not getting shot.
\begin{description}

\item[Defending/Engaged defensive] I'm busy maneuvering to avoid getting shot by
    a missile. Provide additional information if you can.

    ``Defending \designation{SA-5}, West!''

\item[Delouse] A hostile jet is chasing me; please shoot them!

\item[Dirt] A \acronym{sam} radar appeared on my \acronym{rwr}.

\item[Fox 1] I'm shooting a semi-active radar homing \acronym{(sarh)}
    missile, like an \designation{AIM-7} Sparrow.
\item[Fox 2] I'm shooting a heat-seeking missile, like an
    \designation{AIM-9} Sidewinder.
\item[Fox 3] I'm shooting an active radar homing \acronym{(arh)} missile,
    like an \designation{AIM-120} AMRAAM.

\item[Guns] I'm shooting my cannon.

\item[Nails] An aircraft's search radar appeared on my \acronym{rwr}.

\item[Magnum] I'm shooting an anti-radiation missile,
    like an \designation{AGM-88} HARM.

\item[Paveway] I'm dropping a laser-guided bombs,
    like a \designation{GBU-}\designation{10/12/16}

\item[Pickle] I'm dropping bombs (usually unguided),
    like the \designation{Mark 80} series.

\item[Pig] I'm dropping a glide weapon, like a \designation{AGM-154} JSOW.

\item[Rifle] I'm shooting an air-to-ground missile,
    like an \designation{AGM-65} Maverick.

\item[SAM] I \emph{see} an incoming surface to air missile!

\item[Singer] My \acronym{rwr} indicates a \acronym{sam} is shooting at me!

\item[Slapshot] Shoot an anti-radiation missile at the given radar/direction \\
    (as fast as possible, because \emph{they're shooting at me!})

    ``Slapshot \designation{SA-3}, South!''

\item[Sniper] Shoot an anti-radiation missile at the given \acronym{sam} radar.

    ``Sniper \designation{SA-2}, Bullseye 270 30.''

\item[Spike/Mud Spike] An [aircraft/\acronym{sam}] radar is tracking me. \\
    (Now might be a good idea to start defending.)

\item[Splash] Target destroyed. Sometimes \textbf{Shack} is used for
    air-to-ground impacts.

\end{description}
