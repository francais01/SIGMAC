\chapter{Air to Air}

Intercepting and shooting other airplanes is about managing \emph{time and space}.
If you know a bandit is 100 miles away from your jet,
you have lots of time to create, execute, and evaluate a plan.
If someone pops-up five miles away and starts shooting at you,
all you can do is react.
Situational awareness \acronym{(sa)} is king---knowing what's going on around
you will keep you alive and let \emph{you} decide how a fight will unfold.

\section{BVR: Searching}

You've got to find the bad guys before you can shoot them,
so use the tools you have to build a \introduce{picture} of the airspace
around you.
These tools include your own radar, as well as
airborne or ground-based command and control \acronym{(c2)},
datalink, and other planes in your flight.
Communicate and share information with each other.\punckern\footnote{See
\chapref{brevity} for more on how to quickly communicate.}

Searching with your radar is something that comes with practice.
While details vary from plane to plane,
all of them\footnote{Ignoring very modern active electronically scanned array
\acronym{(aesa)} radars, which you probably won't find in a sim.}
have the same fundamental limitation: they work by pointing a narrow radar beam
around using an antenna on a gimbal in your nose.
This means you need to choose between searching a big area slowly,
or searching a smaller area rapidly.
You change the size and shape of this area by controlling the horizontal azimuth
and the number of vertical bars in your scan.
Imagine searching a dark room with a narrow flashlight beam.
You can examine the whole room by swinging the flashlight in wide arcs
in front of you (azimuth), and repeating those arcs up and down (bars).
Or you can focus on a particular corner by swinging the light in short arcs
repeated over fewer vertical motions.
There's no ``right'' approach---it depends on the situation.

\section{BVR engagements}

You've found something you want to shoot. Great!
Next make sure to \emph{positvely identify}
that it's worth shooting with whatever you have at your disposal:
\acronym{iff} transponders,
declarations from \acronym{c2},
non-cooperative-target recognition \acronym{(nctr)}

Once you've found some contacts you're interested in,
use radar modes like track-while-scan \acronym{(tws)} to follow the target.
This will provide you with frequent updates on the target's position, heading,
and speed, while still allowing you to search the space around them.
From here, 
