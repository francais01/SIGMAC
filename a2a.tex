\chapter{Air to Air: Beyond Visual Range}

Intercepting and shooting other airplanes is about managing \emph{time and space}.
If you know a bandit is 100 miles away from your jet,
you have lots of time to create, execute, and evaluate a plan.
If someone pops up five miles away and starts shooting at you,
all you can do is react.
Situational awareness \ac{(sa)} is king---knowing what's going on around
you will keep you alive and let \emph{you} decide how a fight will unfold.

\section{Searching}

You've got to find the bad guys before you can shoot them,
so use the tools you have to build a \introduce{picture} of the airspace
around you.
These tools include your own radar,
airborne or ground-based command \& control \ac{(c2)},
datalink, and other planes in your flight.
Communicate and share information with each other.\punckern\footnote{See
\chapref{brevity} for more on how to quickly communicate.}

Using your radar to search for targets comes with practice.
While details vary from plane to plane,
all of them\footnote{Ignoring very modern active electronically scanned array
\ac{(aesa)} radars, which you probably won't find in a sim.}
work by moving a narrow radar beam around by pointing an antenna in your nose.
The narrow field of view of this antenna means you need to choose between
slowly searching a big area, or rapidly searching a small area.

Imagine you're in a dark room with a flashlight.
You can slowly reveal the whole room by swinging the flashlight in wide arcs
in front of you (increasing your scan azimuth),
and repeating those arcs up and down, one on top of each other
(scan bars).
Or you can focus on a particular corner by swinging the light in short arcs
repeated over fewer vertical bars.
There's no ``right'' search settings---it
all depends on the situation.

Once you've found some contacts you're interested in,
use radar modes like track-while-scan \ac{(tws)} to follow the target.
This will provide you with frequent updates on the target's position, heading,
and speed, while still providing \ac{sa} of the surrounding airspace.

\section{IFF}

You've found something you want to shoot. Great!
Next make sure to \emph{positvely identify}
that it's worth shooting using tools like
\ac{iff} transponders,
declarations from \ac{c2},
and non-cooperative-target recognition \ac{(nctr)}.
Once you're sure they're hostile, let's talk about shooting missiles.

\section{Missile kinematics and you}

At the risk of stating the obvious, you win a \ac{bvr} fight by making sure your
missile hits them before their missile hits you.
But outside the movie \emph{Behind Enemy Lines},
missiles don't endlessly chase down their targets.
The rocket motor burns for a few seconds,
accelerating the missile to very high speeds (Mach 2+),
then the missile coasts to its target.\punckern\footnote{Many
\ac{bvr} missiles also use the time the motor burns to climb---or
\introduce{loft}---to higher altitudes where there is less air resistance.}

This makes \ac{bvr} a game of \emph{energy} management.
You want to maximize your missile's energy so it can get to your target.
At the same time, you want minimize any incoming missile's energy
so that it falls out of the sky trying to catch you.
Let's talk about the ``not getting shot'' part first.

\textbf{\textsc{Do not fly directly towards someone shooting
you.}}\punckern\footnote{Or who \emph{might} be shooting you---we'll talk about
active radar homing \ac{(arh)} missiles shortly.}
Doing so \emph{maximizes} an incoming missile's energy,
since it just has to fly in a straight line, right into your dumb face.
Instead, we're going to \introduce{crank}.
Once an enemy gets close enough that they could feasibly shoot you,
turn so that they're on the side of your radar scope,
a few degrees from its gimbal limit.
This does two important things:
first, it maximizes the distance an incoming missile has to fly to get you
(while still allowing you to track them with your radar).
Second, it forces an incoming missile to \emph{continuously turn} to
aim at you.
And because missiles, like airplanes, turn by sticking their control surfaces
into the wind stream, this slows them down.
Speed is also a factor---the faster you're going,
the further a missile has to chase you down.

If you're up against a competent adversary and cranking alone isn't enough to
defeat an incoming missile, it's time for more drastic measures.
Dive to lower altitudes, where the thicker air will induce more drag on
the missile. Fly a perpendicular heading to the missile---\introduce{beam} it.
Beaming maximizes the amount the missile has to turn,
and makes you hard to track on radar,
especially if you're below the one locking you up.
Dispense chaff and flares as-needed.
Finally, when the missile is close,\punckern\footnote{Knowing \emph{when} a
missile is seconds from impact comes with timing, experience, and a bit of luck.}
pull a high-G turn \emph{into} the missile and across its nose.
To cut the corner, the missile will have to turn even harder,
bleeding lots of energy.

All the same factors apply when you're the shooter.
The time to squeeze the trigger depends on:
\begin{itemize}
\item Range (obviuosly)
\item Altitude: higher is better, but being too high can make it hard to
    maneuver if you get shot at.
\item Aspect \& closure rate:
    A missile has less distance to cover if you and your target are racing
    towards each other than if the target is running away.
\end{itemize}
Once you've locked onto a target in,
your jet's avionics will calculate a few ranges and display them on your
heads-up display \ac{(hud)}.
These usually include:
\begin{itemize}
\item The maximum aerodynamic range of the missile, usually called
    \fakesub{R}{aero}.
    A shot at this range will only hit if the target doesn't maneuver
    \emph{at all} after you launch.
\item The maximum range where a hit is \emph{probable},
    assuming the target continues on its current course at its current airspeed.
    This is usually called \fakesub{R}{max} or \fakesub{R}{pi}.
\item The maximum range where a hit is probable
    even if the target turns and runs away at its current airspeed
    as soon as you launch.
    This is usually called \fakesub{R}{tr} (Range, turn \& run)
    or \fakesub{R}{ne} (Range, no escape).
\item The minimum range where the missile has enough room to turn into the
    enemy and arm itself, called \fakesub{R}{min}.
\end{itemize}
Notice that all of these ranges are expressed as probabilities!
Even if the missile works perfectly,
you don't know how your target will start maneuvering once you've shot at them.
A shot inside \fakesub{R}{ne} could still be defeated if the enemy can accelerate,
break your radar lock, or outmaneuver the missile as it gets close.
Holding your shot (above \fakesub{R}{min}) will almost always give you
a better chance at a kill,
but it also increases the likelyhood you'll get hit in the process.

Keeping all that in mind, let's talk tactics.

\section{The SARH fight}

Semi-active radar homing \ac{(sarh)} missiles like the \designation{AIM-7}
lack their own radar emitter, so to home in on your target,
you must continuously ``paint'' it with your radar in single-target track
\ac{(stt)} mode.
This presents two unique challenges:
\begin{enumerate}
\item Any maneuvering you do must keep the target inside the limits of your radar
    gimbals. In other words,
    you have to keep flying towards the target until your missile hits.
\item Since your radar is only pointing at the target in \ac{stt} mode,
    so you lose the \ac{sa} it would normally give you.
\end{enumerate}

\section{The ARH fight}

Active radar homing \ac{(arh)} missiles like the \designation{AIM-120} AMRAAM
make your job easier for one reason: you can turn around.

\chapter{Dogfighting}
