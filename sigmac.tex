% Build this document with LuaTeX, a modern Unicode-aware LaTeX engine
% that uses system TTF and OTF font files.
% This is needed for the fontspec, microtype, and nolig packages.
%
% We're using KOMA Script to hand-tune footnotes and TOC appearance.
% It should be available in your texlive distribution,
% which is how most distros package LaTeX.
\documentclass[fontsize=10bp, numbers=endperiod, draft=true]{scrbook}

% Margins: see http://practicaltypography.com/page-margins.html and
% http://practicaltypography.com/line-length.html
% We're aiming for 80-ish characters per line.
\usepackage[
    a5paper,
    inner=0.6in,outer=0.6in,top=0.75in,bottom=0.5in,
    footnotesep=10bp,
    footskip=3em,
    includefoot,
]{geometry}

\usepackage[fleqn]{amsmath}

\usepackage{fontspec}

\setmainfont[
    Ligatures=TeX,
    Numbers=Lowercase,
    SmallCapsFeatures={StylisticSet=10},
    UprightFeatures={
        SizeFeatures={
            {Size={-10},Font="Century Supra T4", SmallCapsFont="Century Supra C4"},
            {Size={10-},Font="Century Supra T3", SmallCapsFont="Century Supra C3"},
        },
    },
    BoldFeatures={
        SizeFeatures={
            {Size={-10},Font="Century Supra T4 Bold", SmallCapsFont="Century Supra C4 Bold"},
            {Size={10-},Font="Century Supra T3 Bold", SmallCapsFont="Century Supra C3 Bold"},
        },
    },
    ItalicFeatures={
        SizeFeatures={
            {Size={-10},Font="Century Supra T4 Italic", SmallCapsFont="Century Supra C4"},
            {Size={10-},Font="Century Supra T3 Italic", SmallCapsFont="Century Supra C3"},
        },
    },
    BoldItalicFeatures={
        SizeFeatures={
            {Size={-10},Font="Century Supra T4 Bold Italic", SmallCapsFont="Century Supra C4 Bold"},
            {Size={10-},Font="Century Supra T3 Bold Italic", SmallCapsFont="Century Supra C3 Bold"},
        },
    },
]{Century Supra T3}

\setsansfont[Ligatures=TeX,
    StylisticSet=01,
    UprightFont = *-400-Book,
    ItalicFont = *-400-BookItalic,
    BoldFont = *-600-Semi,
    BoldItalicFont = *-600-SemiItalic
]{Jost}

\setmonofont[
    Numbers=SlashedZero,
    Scale=MatchLowercase,
    UprightFeatures={
        SizeFeatures={
            {Size={-10}, Font=DriveMono-Book},
            {Size={10-}, Font=DriveMono-Regular},
        },
    },
    ItalicFeatures={
        SizeFeatures={
            {Size={-10}, Font=DriveMono-BookItalic},
            {Size={10-}, Font=DriveMono-Italic},
        },
    },
    BoldFont=DriveMono-Bold,
    BoldItalicFont=DriveMono-BoldItalic
]{Drive Mono}

\usepackage{polyglossia}
\setdefaultlanguage[variant=american]{english}

\usepackage{microtype} % Font expansion, protrusion, and other goodness

% Disable ligatures across grapheme boundaries
% (see the package manual for details.)
\usepackage[american]{selnolig}
% I have no idea why Garamond has this on by default.
\nolig{Th}{T|h}

% Use symbols for footnotes, resetting each page
\usepackage[perpage,bottom,symbol*]{footmisc}

% Left flush footnotes. See the KOMA Script manual.
\deffootnote[1em]{1em}{1em}{\thefootnotemark}
% Set the width of the rule separating body text and footnotes
\setfootnoterule{0.8\textwidth}

% Like many fonts, Century Supra's asterisk is already set in a "superscripted" form.
% Superscripting *that* makes it annoyingly small.
% To fix this, we have to redefine footnote marks so that they aren't superscript,
% then raise all the other symbols.
%
% Feel free to remove this if your body type doesn't have this peculiarity,
% but unfortunately many do.
% See http://tex.stackexchange.com/a/16241
%
% We use the Unicode symbols themselves (instead of \dagger, \ddagger, \P, etc.)
% because the latter fall back to Computer Modern/Latin Modern in some cases,
% (e.g., if you're using mathastext instead of unicode-math).
% Alternatively, you could use \textdagger, \textddagger, etc.,
% but this seems more concise.
\DefineFNsymbols*{tweaked}{%
    {*}%
    {\textsuperscript†}%
    {\textsuperscript‡}%
    {\textsuperscript{◊}}%
    {\textsuperscript{§}}%
    {**}%
    {\textsuperscript{††}}%
    {\textsuperscript{‡‡}}%
}
\setfnsymbol{tweaked}
\deffootnotemark{\thefootnotemark}

\DeclareTOCStyleEntry[%
    beforeskip=8bp,
    entrynumberformat = \sffamily\addfontfeature{Numbers=Tabular},
    pagenumberformat = \sffamily\addfontfeature{Numbers=Tabular},
    linefill = \TOCLineLeaderFill
]{tocline}{chapter}
\DeclareTOCStyleEntry[%
    beforeskip=1bp,
    entrynumberformat = \sffamily\addfontfeature{Numbers=Tabular},
    pagenumberformat = \sffamily\addfontfeature{Numbers=Tabular},
    indent=0.4in
]{tocline}{section}
\DeclareTOCStyleEntry[%
    beforeskip=0pt,
    entrynumberformat = \sffamily\addfontfeature{Numbers=Tabular},
    pagenumberformat = \sffamily\addfontfeature{Numbers=Tabular},
    indent=0.6in
]{tocline}{subsection}

% Don't use a sans font for description labels.
\addtokomafont{descriptionlabel}{\rmfamily}
% Sections and such use serif type.
\setkomafont{disposition}{\sffamily}
% Uppercase numbers for chapters.
% Set size and style of section and subsections
\setkomafont{section}{\Large\itshape}
\setkomafont{subsection}{\large\itshape}
% Don't put big spaces before chapter headings
\renewcommand{\chapterheadstartvskip}{\vspace{0.15in}}

\setcapwidth[c]{.75\textwidth}
%\setcapmargin{0pt}
\setkomafont{caption}{\sffamily\footnotesize}
\setkomafont{captionlabel}{\sffamily\footnotesize}
\renewcommand*{\figureformat}{}
\renewcommand*{\tableformat}{}
\renewcommand*{\captionformat}{}

% Use uppercase numbers for numbered lists.
% (We're using lowercase ones for the body text.)
% See http://tex.stackexchange.com/a/133186
\usepackage{enumitem}
\setlist[enumerate]{font=\addfontfeatures{Numbers=LowercaseOff}}
\setlist[description]{leftmargin=1em}

\usepackage{tabularx}

% Custom footer
\usepackage[draft=false]{scrlayer-scrpage}
\clearpairofpagestyles
\pagestyle{scrheadings}
\setkomafont{pagefoot}{\upshape\sffamily}
\cefoot*{\thepage}
\cofoot*{\thepage}

\usepackage{changepage} % For adjustwidth

\usepackage{metalogo} % for \LuaLaTeX

% Indent code examples, etc., by double the text size.
\newenvironment{leftfigure}
  {\par\vspace{0.5\baselineskip minus 0.3\baselineskip}\begin{adjustwidth}{22bp}{0pt}}
  {\end{adjustwidth}\vspace{0.5\baselineskip minus 0.3\baselineskip}}

% Like the above, but with no adjustwidth
\newenvironment{flushleftfigure}
  {\par\vspace{0.5\baselineskip minus 0.3\baselineskip}\noindent\ignorespacesafterend}
  {\vspace{0.5\baselineskip minus 0.3\baselineskip}\par\noindent\ignorespacesafterend}

\newenvironment{centerfigure}
  {\par\vspace{0.5\baselineskip minus 0.3\baselineskip}\begin{adjustwidth}{22bp}{22bp}\centering}
  {\end{adjustwidth}\vspace{0.5\baselineskip minus 0.3\baselineskip}}

\usepackage{graphicx}

\usepackage{csquotes}

\newcommand{\edition}{First edition \acronym{(WIP)}}

\title{Some Idiot's Guide to Modern Air Combat}
\author{José Jimémenez}
\date{\today}

% Custom footer
% Hyperlinks
\usepackage[unicode,pdfusetitle,draft=false]{hyperref}
\hypersetup{
    colorlinks=true, % Use colors
    linkcolor=black, % Intra-doc links are black, like the rest of the text.
    urlcolor=blue % URLs are blue
}

% Use \punckern to overlap periods, commas, and footnote markers
% for a tighter look.
% Care should be taken to not make it too tight - f" and the like can overlap
% if you're not careful.
\newcommand{\punckern}{\kern-0.2ex}
% For placing commas close to, or under, quotes they follow.
% We're programmers, and we blatantly disregard American typographical norms
% to put the quotes inside, but we can at least make it look a bit nicer.
\newcommand{\quotekern}{\kern-0.5ex}


% Create an unbreakable string of text in a monospaced font.
% Useful for `command --line --args`
\newcommand{\monobox}[1]{\mbox{\texttt{#1}}}

% Italicize new terms
\newcommand{\introduce}[1]{\textit{#1}}

% Letterspace acronyms a bit.
\newcommand{\acronym}[1]{\textsc{#1}}

\newcommand{\designation}[1]{\mbox{\addfontfeature{Numbers=LowercaseOff}#1}}

% "Chapter <num>" references
\newcommand{\chapref}[1]{chapter~\ref{#1}}

% monospace URLs (without setting the http://...)
\newcommand{\http}[1]{\url{http://#1}}
\newcommand{\https}[1]{\url{https://#1}}

% See http://tex.stackexchange.com/a/68310
\makeatletter
\let\runauthor\@author
\let\rundate\@date
\let\runtitle\@title
\makeatother

% Spend a bit more time to get better word spacing.
% See http://tex.stackexchange.com/a/52855/92465
\emergencystretch=1ex

% Do as I say, not as I do.
\widowpenalty=10000
\clubpenalty=10000

% For an online PDF, blank pages are annoying.
\renewcommand{\cleardoublepage}{\clearpage}

\begin{document}
\fontsize{10bp}{13bp}\selectfont

\frontmatter
\setcounter{secnumdepth}{0}
\setlength\parindent{0pt}

% Custom title instead of \maketitle
\pagenumbering{gobble}
\vspace*{1in}
\begin{center}
\sffamily
\fontsize{0.25in}{0.3in}\selectfont
Some Idiot's Guide to \\
\fontsize{0.5in}{0.6in}\selectfont
Modern Air Combat

\normalsize
\vspace{1.5\baselineskip}
\edition
\vspace{2in}

\LARGE
\runauthor
\end{center}
\clearpage

{\raggedright%For the page
\null
\vfill

Copyright © 2018--2020 \\
by \runauthor
\bigskip

This book is licensed under the \\
{\addfontfeature{Numbers={LowercaseOff}}Creative Commons Attribution-ShareAlike~4.0}
International License. \\
In short, you are free to share, translate, adapt, or improve this book
so long as you give proper credit and provide your contributions under
the same license. \\
The license's full text is available at \\
\https{creativecommons.org/licenses/by-sa/4.0/legalcode}

\vspace{0.5in}
\edition{} (online \acronym{pdf}), typeset \today.
} % end ragged right
\clearpage

\vspace*{1in}
{\itshape%
Some dedication here.
}
\cleardoublepage

\pagenumbering{roman}
\tableofcontents

\mainmatter
% Indent by one lead, as suggested in The Elements of Typographic Style.
\setlength\parindent{14bp}

\pagenumbering{arabic}
\setcounter{page}{1} % Restart page numbering after the ToC.
\cleardoublepage

\refoot*{SIGMAC}
\rofoot*{SIGMAC}

\chapter{What?}

If you're getting into a modern combat flight sim like BMS or DCS,
there's lots of great guides to learn how to fly:
\href{https://www.mudspike.com/chucks-guides/}{Chuck's},
\href{https://www.youtube.com/channel/UCvXXUrGCF3wV3bbZ6pFQ00g}{Jabbers},
\href{https://www.youtube.com/user/RedKiteRender}{RedKite},
and
\href{https://www.youtube.com/channel/UCqH078Ef0HENo01LF3xwIvA}{Crash Laobi},
just to name a few.
They'll show you how to start up, take off, fly around
without ripping your wings off, shoot some weapons, and land.

But once you've got a decent grasp on how to fly a jet,
guides for \emph{employing} that jet seem harder to come by.
Once you know how not to crash the thing, you probably want to learn how to:
\begin{itemize}
\item Dogfight
\item Shoot other planes beyond visual range \ac{(bvr)}
\item Defend yourself from a surface-to-air missile \ac{(sam)}
\item Put a bomb on a target
\item Act as a wingman or a flight lead, and fly in formation.
\item Communicate with other pilots, and sound cool doing so
\end{itemize}
% \dots all without taking a missile to the face.

Specific tactics vary from era to era and jet to jet,
but many of the same fundamentals have held for the last 50
years.\punckern\footnote{We'll try to discuss how evolving technology
enables newer tactics as we go.}
This won't be a comprehensive guide,
and contributions are welcome.
But it should outline the basics, and point you to resources where you can
learn more.

\section{On Alphabet Soup}

Militaries and pilots both love acronyms,
and the jargon you get when you cross the two is an ungodly mess.
I'll try to introduce relevant acronyms and terms where I can,
but I'm bound to miss a few.
Thankfully, the Internet is full of useful aids---hopefully they can help you
out.\punckern\footnote{%
\href{https://www.fighterpilotpodcast.com/glossary/}{The Fighter Pilot Podcast's glossary}
is a great example.}

\chapter{Brevity}
\label{brevity}

Like most professions, combat aviation has its own jargon.
Called \emph{brevity}, it's designed to let pilots communicate
as quickly as possible in the heat of battle.

Most folks you find online will be using \textsc{us/nato} brevity.
There's many,
many terms,\punckern\footnote{An early-2000s version of an official
multiservice \textsc{us} training guide is declassified and freely available
online at \url{https://apps.dtic.mil/dtic/tr/fulltext/u2/a404426.pdf}}
but we'll cover some of the most common ones.
Before we dive in, though, a few points:
\begin{enumerate}
\item Basic radio protocol is, ``<you>, <me>. <what I have to say>''
    For example, if my callsign is Warhawk and I want to ask another flight,
    callsign Hammer, what they're up to, I'd say,
    ``Hammer, Warhawk. Status''

    Of course, this isn't an absolute rule.
    If I'm leading a flight and I want us to get ready for combat, I'll say,
    ``Flight, fence in''\quotekern, not, ``Flight, lead. Fence in.''
    Similarly, my wingman will just say, ``Two is fenced in\dots''
    not, ``Lead, two, fenced in\dots''
    Don't be an annoying pedant.

\item If you can't think of the right brevity,
    \emph{use plain language!}
    The main goal is to be understood, quickly.
    Sounding cool is an added bonus.

\item Provide additional information whenever you can.
    If you're shooting an AMRAAM, ``Fox~3'' isn't a very helpful call---it tells
    me a missile is coming off your jet, but it could be pointed at my ass
    for all I know.
    ``Fox~3, Bullseye 069 42'' lets me know what the intended target is,
    and I'll maneuver (and engage other targets!) to best support you.
    Similarly, if the two of us merge with a pair of bandits,
    and you call ``Fox 2, \designation{Mig-21} low''\quotekern,
    I'll take the high one.

    Work with friendlies to build a shared understanding of what's going on
    around you.

\item Some terms have one version for friendly forces
    and another for hostiles or unknown forces.
    Don't mix them up!
    If you're \emph{visual} on your wingman,
    you can fly in formation with them.
    But if you \emph{tally} your wingman,
    they'll wonder why you're trying to shoot them.
\end{enumerate}

\section{Who? Where?}

\begin{description}
\item[BRAA] Short for \emph{Bearing, Range, Altitude, Aspect}.
    From my plane, the thing I'm describing is in this direction
    (bearing), for this distance (range), at this altitude.
    Provide the direction they're moving (aspect) if possible.

    ``Bogey, BRAA 035 50, 18 thousand, hot.''

    Bearings are in degrees and ranges are in miles.
    Altitudes are in thousands of feet, called ``thousand'' for hostiles
    and ``angels'' for friendlies. Aspects include:
    \begin{description}
    \item[Hot] heading towards you
    \item[Dragging] heading away from you
    \item[Flanking] approaching you at an angle (provide a direction)
    \item[Beaming] heading perpendicular to you (provide a direction)
    \end{description}

\item[Bogey Dope] Tell me where the closest threat is.
    Requested from command and control \ac{(c2)} assets,
    like an airborne \designation{E-3} Sentry
    or a ground-based fighter controller.

\item[Bullseye] The thing I'm describing is this direction and distance
    from the \emph{bullseye}.
    The bullseye is a preplanned location everyone has agreed to use for
    a reference. It could be a large ground target (like an airfield),
    some landmark, or just an arbitrary point---all that matters is
    everyone knows where it is.

    ``Engaging armored column, Bullseye 180 20''

\item[Capture] I see a ground target on my sensors (such as radar
    or targeting pod).

\item[Contact] I see a potential target on my sensors (usually radar).
    Also brevity for the potential target itself.
    Follow with a location.

    ``Contact, Bullseye 120 35, 18 thousand. Hot.''

\item[Declare] Tell me if the contact at the given location
    (using BRAA or a Bullseye call) is friendly or not (see below).
    Requested from \ac{c2} assets.

\item[Friendly/Bogey/Outlaw/Bandit/Hostile]
    In order of increasing threat:
    \begin{description}
    \item[Friendly] An positively-identified allied contact.
    \item[Bogey] An unknown contact.
    \item[Outlaw] A bogey that is likely a hostile
        (e.g., that launched from an enemy airfield)
    \item[Bandit] An enemy contact---does not imply permission to engage.
    \item[Hostile] An enemy contact---free to fire.
    \end{description}

\item[Furball] A dogfight of friendlies and hostiles.
    Don't shoot an active radar missile into these or it might go pitbull
    on a friendly.

\item[Group] Contacts flying together.

\item[Merged] Contacts---usually bandits---are within visual range
    \ac{(wvr)}.
    Get your eyes out of the cockpit and start turning \& burning.
    Named after what happens to multiple contacts when they run into each other
    on someone's radar screen---they \emph{merge}.

\item[Picture] Tell me about nearby contacts (with BRAA or Bullseye calls).
    Requested from \ac{c2} assets.

\item[Pop-up] A new contact has suddenly appeared at the given location.

\item[Tally/No Joy] I see the enemy.\,/\,I don't see the enemy.

\item[Tumbleweed] I've lost situational awareness \ac{(sa)}
    and don't know what's going on.
    Help me get reoriented, if you can.

\item[Visual/Blind] I see the friendly.\,/\,I don't see the friendly.
\end{description}

\section{Administration}

What are you doing? What's your status?
\begin{description}

\item[As Fragged] I'm continuing as-planned.

\item[Bingo] I'm low on fuel and need to go home right now if I want enough
    to land.

\item[Commit] Intercept the given group.

    You usually brief a \emph{commit range}---if contacts get within that range,
    you stop what you're doing and commit on those contacts.

\item[Fence In] Get ready for combat---shut off external lights,
    arm weapons, enable electronic countermeasures \ac{(ecm)}, etc.
    From ``crossing the fence'' as you enter bad guy territory.
    Once you're out of combat, you \emph{fence out}.

\item[Gate/Buster] Fly as fast as you can [with/without] afterburner.

\item[Joker] I'm starting to run low on fuel and should think about getting
    out of here. Comes before bingo.

\item[Rolex] Delay the (package's) plan by the given amount of time.

    ``Package, rolex plus three [minutes].''
\end{description}

\section{Combat}

Shooting things and not getting shot.
\begin{description}

\item[Defending/Engaged defensive] I'm busy maneuvering to avoid getting shot by
    a missile. Provide additional information if you can.

    ``Defending \designation{SA-5}, West!''

\item[Delouse] A hostile jet is chasing me; please shoot them!

\item[Dirt] A \ac{sam} radar appeared on my \ac{rwr}.

\item[Fox 1] I'm shooting a \ac{sarh} missile,
    like an \designation{AIM-7} Sparrow.
\item[Fox 2] I'm shooting a heat-seeking missile, like an
    \designation{AIM-9} Sidewinder.
\item[Fox 3] I'm shooting an \ac{arh} missile,
    like an \designation{AIM-120} AMRAAM.

\item[Guns] I'm shooting my cannon.

\item[Nails] An aircraft's search radar appeared on my \ac{rwr}.

\item[Magnum] I'm shooting an anti-radiation missile,
    like an \designation{AGM-88} HARM.

\item[Paveway] I'm dropping a laser-guided bombs,
    like a \designation{GBU-}\designation{10/12/16}

\item[Pickle] I'm dropping bombs (usually unguided),
    like the \designation{Mark 80} series.

\item[Pig] I'm dropping a glide weapon, like a \designation{AGM-154} JSOW.

\item[Rifle] I'm shooting an air-to-ground missile,
    like an \designation{AGM-65} Maverick.

\item[SAM] I \emph{see} an incoming surface to air missile!

\item[Singer] My \ac{rwr} indicates a \ac{sam} is shooting at me!

\item[Slapshot] Shoot an anti-radiation missile at the given radar/direction \\
    (as fast as possible, because \emph{they're shooting at me!})

    ``Slapshot \designation{SA-3}, South!''

\item[Sniper] Shoot an anti-radiation missile at the given \ac{sam} radar.

    ``Sniper \designation{SA-2}, Bullseye 270 30.''

\item[Spike/Mud Spike] An [aircraft/\ac{sam}] radar is tracking me. \\
    (Now might be a good idea to start defending.)

\item[Splash] Target destroyed. Sometimes \textbf{Shack} is used for
    air-to-ground impacts.

\end{description}


\setlength\parskip{0.55\baselineskip}
\setlength\parindent{0pt}

\backmatter

\chapter{Colophon}

This guide was typeset with \LuaLaTeX{} in Matthew Butterick's
Century Supra,
<more here> and \sffamily Jost* <more here>.


\end{document}
